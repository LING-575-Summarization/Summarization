% This must be in the first 5 lines to tell arXiv to use pdfLaTeX, which is strongly recommended.
\pdfoutput=1
% In particular, the hyperref package requires pdfLaTeX in order to break URLs across lines.

\documentclass[11pt]{article}

% Remove the "review" option to generate the final version.
\usepackage[]{style/acl2023}

% Standard package includes
\usepackage{times}
\usepackage{latexsym}
\usepackage{amsmath}
\usepackage{amssymb}
\usepackage{amsthm}
\usepackage{relsize}
\usepackage{dsfont}
\usepackage{graphicx}
\usepackage{float}
% For proper rendering and hyphenation of words containing Latin characters (including in bib files)
\usepackage[T1]{fontenc}
% For Vietnamese characters
% \usepackage[T5]{fontenc}
% See https://www.latex-project.org/help/documentation/encguide.pdf for other character sets

% This assumes your files are encoded as UTF8
\usepackage[utf8]{inputenc}

% This is not strictly necessary, and may be commented out.
% However, it will improve the layout of the manuscript,
% and will typically save some space.
\usepackage{microtype}

% This is also not strictly necessary, and may be commented out.
% However, it will improve the aesthetics of text in
% the typewriter font.
\usepackage{inconsolata}

%%%%%%%%%%%%%%%%%%%%%%%%%%%%%%%%%%%%%%%%%%%
%%%%%% Specify Custom Packages below %%%%%%
%%%%%%%%%%%%%%%%%%%%%%%%%%%%%%%%%%%%%%%%%%%

\usepackage{hyperref}
\usepackage{bm}

\usepackage{algorithm}
\usepackage[noend]{algpseudocode}
\usepackage{tabularx}

% \usepackage{stfloats}
\usepackage{indentfirst}


% If the title and author information does not fit in the area allocated, uncomment the following
%
%\setlength\titlebox{<dim>}
%
% and set <dim> to something 5cm or larger.

\title{D1}

% Author information can be set in various styles:
% For several authors from the same institution:
\author{Anna Batra, Sam Briggs, Junyin Chen, Hilly Steinmetz\\
          Department of Linguistics, University of Washington \\
          \texttt{\{batraa, briggs3, junyinc, hsteinm\}@uw.edu}}
% if the names do not fit well on one line use
%         Author 1 \\ {\bf Author 2} \\ ... \\ {\bf Author n} \\
% For authors from different institutions:
% \author{Author 1 \\ Address line \\  ... \\ Address line
%         \And  ... \And
%         Author n \\ Address line \\ ... \\ Address line}
% To start a seperate ``row'' of authors use \AND, as in
% \author{Author 1 \\ Address line \\  ... \\ Address line
%         \AND
%         Author 2 \\ Address line \\ ... \\ Address line \And
%         Author 3 \\ Address line \\ ... \\ Address line}

% \author{First Author \\
%   Affiliation / Address line 1 \\
%   Affiliation / Address line 2 \\
%   Affiliation / Address line 3 \\
%   \texttt{email@domain} \\\And
%   Second Author \\
%   Affiliation / Address line 1 \\
%   Affiliation / Address line 2 \\
%   Affiliation / Address line 3 \\
%   \texttt{email@domain} \\}

\begin{document}
\maketitle
\begin{abstract}
This deliverable contains a skeleton of the paper and a list of team members. This is an example citation \citep{jurafsky-martin-draft}. According to \citet{jurafsky-martin-draft}, this is another example citation.
\end{abstract}

\section{Introduction}

\section{Engines}

\section{System Overview}

\section{Approach}

\section{Results}

\section{Discussion}

\section{Conclusion}

\section{Appendix A: Workload distribution}

\begin{itemize}
    \item Anna Batra set up the Github repository, turned in D1
    \item Junyin Chen got the team together and set up a communication channel
    \item Sam Briggs set up the Overleaf file and sent out a when-to-meet to schedule weekly meetings
    \item Hilly Steinmetz edited the Overleaf file to prepare it for D1.
\end{itemize}

\section{Appendix B: Code repository and additional software and data used in your system}

The repository for our project can be found on Github at \href{https://github.com/LING-575-Summarization/Summarization}{github.com/LING-575-Summarization/Summarization}.

We plan to use the following software for the project: 

\begin{itemize}
    \item Python
    \item Anaconda (for virtual environment)
    \item NLTK (for parsing, NER, and more)
    \item spaCy (for parsing, NER, and more)
    \item Scikit-learn (for machine learning models)
    \item Pytorch (for large LMs)
    \item Hugging Face's Transformer library (for auto-tokenization and find-tune trained model)
\end{itemize}

\nocite{jurafsky-martin-draft, radev-etal-2000-centroid}
\bibliography{bibliography/custom}
\bibliographystyle{bibliography/acl_natbib}

\end{document}
