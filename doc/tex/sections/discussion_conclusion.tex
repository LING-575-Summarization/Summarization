\section{Discussion}

\begin{table*}[ht]
\centering
\caption{Generated summaries based on \textbf{devtest} docset D1006}
\begin{tabular}{|l|p{13cm}|}
  \hline
 gold & On Sept 30, Merck voluntarily recalled the pain killer Vioxx, used by almost 2 million, after clinical trials for its use in colon cancer showed unacceptable rates of stroke/heart attack. Results corroborated earlier warnings that had not resulted in recalls by the Food and Drug Administration (FDA). As a COX inhibitor, Vioxx was safer for digestive tracts, important for arthritis patients. Merck's advertising campaigns did not clearly warn about side effects. The case highlighted concerns about drug manufacturers' advertising and FDAs role in insuring safety of drugs on the market. Safety of other COX inhibitors is now a concern." \\
 \hline
ILP D5 & Merck officials said last week its latest research showed an increased risk of heart attack and other cardiovascular complications in patients who took Vioxx for at least 18 months.
Heavily advertised as an arthritis drug, Vioxx was pulled from the market last week after its maker said a study showed it doubled the risk of heart attack and stroke.
But some doctors say this group of drugs may work in a way that increases the risk of heart problems for some patients, and they point to this latest information as additional reason for concern.
 \\
 \hline
ILP D4 & Merck officials said last week its latest research showed an increased risk of heart attack and other cardiovascular complications in patients who took Vioxx for at least 18 months.
Heavily advertised as an arthritis drug, Vioxx was pulled from the market last week after its maker said a study showed it doubled the risk of heart attack and stroke.
But some doctors say this group of drugs may work in a way that increases the risk of heart problems for some patients, and they point to this latest information as additional reason for concern.
 \\
  \hline
 \multicolumn{1}{|p{1.2cm}|}{\centering LexRank \\ (D4)} & With Vioxx, researchers had been warning about the drug's possible cardiovascular risks since 2000, only a year after it was approved by the FDA . Data from a company study found then that users had four times as many heart attacks and strokes as those who used another painkiller . But the data was not definitive, and Merck, which even critics say is one of the most responsible drug companies, repeatedly reassured the medical and financial communities that Vioxx was safe.
 \\
 \hline
 \multicolumn{1}{|p{1.2cm}|}{\centering LexRank \\ (D3)} & \" That's the tragedy here . ''
 And if courts determine that Merck was negligent, the company will pay a heavy price in compensation.
 It is in the insurance industry's interest, the FDA's interest and the federal government's interest--because the federal government is a major provider of health insurance--either to require drug companies to conduct such comparative tests or to set up a neutral agency to do so.
 FDA: http: //www.fda.gov/
 \\
 \hline
\multicolumn{1}{|p{1cm}|}{\centering GSG LLM \\ (D5)} & In September 2004, Merck \& Co. recalled its arthritis drug Vioxx after a clinical trial showed it doubled the risk of heart attacks and strokes. 
 The drug had been used by 20 million Americans since its approval in 1999 and was the company's top-selling product. 
 Merck had spent \$195 million to promote Vioxx as a wonder drug for the aging baby boomers. 
 The FDA, which approved Vioxx for use, had been concerned about the drug's cardiovascular risks since at least 2000 but did not issue a warning until 2004.
 \\
 \hline
\multicolumn{1}{|p{1cm}|}{\centering GSG LLM \\ (D4)} &  Merck recalled Vioxx in September 2004 after a study showed that it doubled the risk of heart attacks and strokes in older people taking it for at least three years. 
 The drug had been approved by the FDA in 1999 for arthritis. 
 Merck had promoted Vioxx as a way to lower blood pressure and cholesterol, but the study showed that it increased the risk of heart attacks and strokes. 
 Merck's decision to withdraw Vioxx from the market raised questions about aggressive marketing of the drug before its long-term safety had been proven.
\\
\hline
\multicolumn{1}{|p{1cm}|}{\centering Baseline \\ Top K \\ (D5)} & Merck Co. said its surprise decision Thursday to withdraw the arthritis drug Vioxx--used by about 2 million people worldwide--was driven by recent evidence that the drug's adverse side effects outweighed any potential benefits.
But that wasn't really news. 
\\
  \hline
\end{tabular}
\label{tab:error_analysis_devtest}
\end{table*}
\begin{table*}[ht]
\centering
\caption{Generated summaries based on \textbf{evaltest} docset D1105}
\begin{tabular}{|l|p{13cm}|}
  \hline
 gold & Boeing 737-400 plane with 102 people on board crashed into a mountain in the West Sulawesi province of Indonesia, on Monday, January 01, 2007, killing at least 90 passengers, with 12 possible survivors. The plane was Adam Air flight KI-574, departing at 12:59 pm from Surabaya on Java bound for Manado in northeast Sulawesi. The plane crashed in a mountainous region in Polewali, west Sulawesi province. There were three Americans on board, it is not know if they survived. The cause of the crash is not known at this time but it is possible bad weather was a factor. \\
 \hline
ILP & An Indonesian lawmaker Tuesday criticised what he said was the slow deployment of search and rescue teams to a mountainous area of Sulawesi island where a plane carrying 102 people crashed.
An Indonesian passenger plane carrying 102 people disappeared in stormy weather on Monday, and rescue teams were sent to search an area where military aviation officials feared the Boeing 737-400 aircraft may have crashed.
An Adam Air Boeing 737-400 plane with 102 people on board crashed in a mountainous area near the town of Polewali late Monday on its way from Surabaya to Manado.
 \\
  \hline
 LexRank & 
 \\
 \hline
 GSG LLM & The Indonesian Navy (TNI AL) has sent two Cassa planes to carry the bodies of five of its members who were killed in a plane crash in Sulawesi late Monday. 
 An Adam Air Boeing 737-400 plane with 102 people on board crashed in a mountainous area near the town of Polewali late Monday on its way from Surabaya to Manado. 
 At least 90 people, including five TNI AL members, were killed in the crash.
 \\
 \hline
 Baseline & Indonesian President Susilo Bambang Yudhoyono said Tuesday he was deeply concerned with the crash of a passenger plane and the sinking of a ferry in the last few days that might have killed hundreds of people.
Earlier on Friday, a ferry carrying 628 people sank off the Java coast with some 400 passengers reported missing.
A Boeing 737-400 plane with 102 people onboard crashed into a mountain in the West Sulawesi province Monday, killing at least 90 people.
JAKARTA, Jan. 2 (Xinhua)
 \\
 \hline
\end{tabular}
\label{tab:error_analysis_evaltest}
\end{table*}
\begin{table*}[t]
    \centering
    \begin{tabular}{|c|p{4.3cm}|p{4.3cm}|p{4.3cm}|}
        \hline
        Docset & TF-IDF & Word2Vec & DistilBert \\
        \hline
        D1001-A & 
        \textcolor{violet}{Graham praised the Columbine community for uniting under the pain of a tragedy that could have torn it apart.}
        
        \textcolor{blue}{But Wells said he is more interested in simply trying to have fun and move beyond the tragedy that put his life on hold.}
        
        \textcolor{purple}{So many forms of community, rippling outward from Columbine High and across the planet, have come together since last week's violence that it was difficult to tell.}
        
        \textcolor{green}{The school wanted to make sure there was enough to eat since students couldn't leave campus for lunch and get back in.}
        & 
        \textcolor{purple}{So many forms of community, rippling outward from Columbine High and across the planet, have come together since last week's violence that it was difficult to tell.}
        
        \textcolor{violet}{Graham praised the Columbine community for uniting under the pain of a tragedy that could have torn it apart.}
        
        \textcolor{green}{The school wanted to make sure there was enough to eat since students couldn't leave campus for lunch and get back in.}
        
        \textcolor{blue}{But Wells said he is more interested in simply trying to have fun and move beyond the tragedy that put his life on hold.}
        & 
        \textcolor{violet}{Graham praised the Columbine community for uniting under the pain of a tragedy that could have torn it apart.}
        
        \textcolor{blue}{But Wells said he is more interested in simply trying to have fun and move beyond the tragedy that put his life on hold.}
        
        \textcolor{purple}{So many forms of community, rippling outward from Columbine High and across the planet, have come together since last week's violence that it was difficult to tell.}
        
        \textcolor{green}{The school wanted to make sure there was enough to eat since students couldn't leave campus for lunch and get back in.}
        \\
        \hline
        D1002-A 
        & 
        \textcolor{red}{Several of the officers are said to have told associates that they continued firing because Diallo did not fall even after they had unleashed the fusillade.}
        
        \textcolor{orange}{They are accused of firing 41 times at Amadou Diallo while searching for a rape suspect on Feb. 4.}
        
        \textcolor{olive}{While the trial date would come nearly a year after Diallo's death on the night of Feb. 4, it is not unusual in such high-publicity cases.}
        
        \textcolor{purple}{Police officers in criminal trials have often asked for a judge to decide their case, fearing that juries would be unsympathetic.}
        
        & 
         \textcolor{orange}{They are accused of firing 41 times at Amadou Diallo while searching for a rape suspect on Feb. 4.}

         \textcolor{purple}{Police officers in criminal trials have often asked for a judge to decide their case, fearing that juries would be unsympathetic.}

        \textcolor{red}{Several of the officers are said to have told associates that they continued firing because Diallo did not fall even after they had unleashed the fusillade.}

        \textcolor{olive}{While the trial date would come nearly a year after Diallo's death on the night of Feb. 4, it is not unusual in such high-publicity cases.}
        & 
        \textcolor{orange}{They are accused of firing 41 times at Amadou Diallo while searching for a rape suspect on Feb. 4.}

        \textcolor{purple}{Police officers in criminal trials have often asked for a judge to decide their case, fearing that juries would be unsympathetic.}

        \textcolor{olive}{While the trial date would come nearly a year after Diallo's death on the night of Feb. 4, it is not unusual in such high-publicity cases.}

        \textcolor{red}{Several of the officers are said to have told associates that they continued firing because Diallo did not fall even after they had unleashed the fusillade.} \\
        \hline
    \end{tabular}
    \caption{Error Analysis for summary ordering using Topic Clustering using \textbf{mean} Fractional Ordering}
    \label{clustering_mean}
\end{table*}
\begin{table*}[h]
    \centering
    \begin{tabular}{|c|p{4.3cm}|p{4.3cm}|p{4.3cm}|}
        \hline
        Docset & TF-IDF & Word2Vec & DistilBert \\
        \hline
        D1001-A & 
        \textcolor{purple}{So many forms of community, rippling outward from Columbine High and across the planet, have come together since last week's violence that it was difficult to tell.}
        
        \textcolor{violet}{Graham praised the Columbine community for uniting under the pain of a tragedy that could have torn it apart.}
        
        \textcolor{blue}{But Wells said he is more interested in simply trying to have fun and move beyond the tragedy that put his life on hold.}
        
        \textcolor{green}{The school wanted to make sure there was enough to eat since students couldn't leave campus for lunch and get back in.}
        & 
        \textcolor{purple}{So many forms of community, rippling outward from Columbine High and across the planet, have come together since last week's violence that it was difficult to tell.}
        
        \textcolor{violet}{Graham praised the Columbine community for uniting under the pain of a tragedy that could have torn it apart.}
        
        \textcolor{green}{The school wanted to make sure there was enough to eat since students couldn't leave campus for lunch and get back in.}
        
        \textcolor{blue}{But Wells said he is more interested in simply trying to have fun and move beyond the tragedy that put his life on hold.}
        & 
        \textcolor{violet}{Graham praised the Columbine community for uniting under the pain of a tragedy that could have torn it apart.}
        
        \textcolor{blue}{But Wells said he is more interested in simply trying to have fun and move beyond the tragedy that put his life on hold.}
        
        \textcolor{purple}{So many forms of community, rippling outward from Columbine High and across the planet, have come together since last week's violence that it was difficult to tell.}
        
        \textcolor{green}{The school wanted to make sure there was enough to eat since students couldn't leave campus for lunch and get back in.}
        \\
        \hline
        D1002-A 
        & 
        \textcolor{red}{Several of the officers are said to have told associates that they continued firing because Diallo did not fall even after they had unleashed the fusillade.}
        
        \textcolor{orange}{They are accused of firing 41 times at Amadou Diallo while searching for a rape suspect on Feb. 4.}
        
        \textcolor{olive}{While the trial date would come nearly a year after Diallo's death on the night of Feb. 4, it is not unusual in such high-publicity cases.}
        
        \textcolor{purple}{Police officers in criminal trials have often asked for a judge to decide their case, fearing that juries would be unsympathetic.}
        
        & 
         \textcolor{orange}{They are accused of firing 41 times at Amadou Diallo while searching for a rape suspect on Feb. 4.}

         \textcolor{purple}{Police officers in criminal trials have often asked for a judge to decide their case, fearing that juries would be unsympathetic.}

        \textcolor{red}{Several of the officers are said to have told associates that they continued firing because Diallo did not fall even after they had unleashed the fusillade.}

        \textcolor{olive}{While the trial date would come nearly a year after Diallo's death on the night of Feb. 4, it is not unusual in such high-publicity cases.}
        & 
        \textcolor{orange}{They are accused of firing 41 times at Amadou Diallo while searching for a rape suspect on Feb. 4.}

        \textcolor{purple}{Police officers in criminal trials have often asked for a judge to decide their case, fearing that juries would be unsympathetic.}

        \textcolor{olive}{While the trial date would come nearly a year after Diallo's death on the night of Feb. 4, it is not unusual in such high-publicity cases.}

        \textcolor{red}{Several of the officers are said to have told associates that they continued firing because Diallo did not fall even after they had unleashed the fusillade.} \\
        \hline
    \end{tabular}
    \caption{Error Analysis for summary ordering using Topic Clustering using \textbf{median} Fractional Ordering}
    \label{clustering_median}
\end{table*}

\subsection{ILP Hyper-parameters}


\begin{table*}[h]
    \centering
    \begin{tabular}{|p{1cm}|p{1cm}|p{1.5cm}|p{1cm}|p{1cm}|p{1cm}|p{1cm}|p{0.7cm}|p{1.5cm}|p{1.5cm}|}
        \hline
         Exp-ID& Min Sent Length& n-gram& delta tf& delta idf& Elim Punc& Lower-casing & log & ROUGE1 & ROUGE2 \\
         \hline
        J0& 25&Unigram&0.01&0.7&No&Yes&Yes&0.33320&0.07410 \\
        J1& None&Unigram&0.01&0.7&No&Yes&Yes&0.32508&0.06849 \\
        J2& 25&Bigram&0.01&0.7&No&Yes&Yes&0.30059&0.07016 \\
        J3& 25&Trigram&0.01&0.7&No&Yes&Yes&0.27682&0.06078 \\
        J4& 26&Unigram&0.001&0.7&No&Yes&Yes&0.33114&0.07349 \\
        J5& 25&Unigram&0.01&0.001&No&Yes&Yes&0.22592&0.03107 \\
        J6& 25&Unigram&0.01&0.7&Yes&Yes&Yes&0.33320&0.07405 \\
        J7& 25&Unigram&0.01&0.7&No&No&Yes&0.33320&0.07405 \\
        J8& 25&Unigram&0.01&0.7&No&Yes&No&0.33320&0.07415 \\
        \hline
    \end{tabular}
    \caption{The results of the experiments that we ran for our hyper-parameter ablation test on ILP. The top row is the top-model with the best combination of hyper-parameters that gets us the our very best ROUGE1 score.}
    \label{ablation}
\end{table*}




We chose the ILP model with hyperparameters as follows as the top model:
\begin{verbatim}
    Min Sent Length = 25
    n-gram = Unigram
    delta tf = 0.01
    delta idf = 0.7
    Eliminate Punctuation = No
    Lower Casing = Yes
    log = Yes
\end{verbatim}

In table \ref{ablation}, we performed an hyper-parameter ablation test\footnote{The ILP hyper-parameter ablation test was performed using the \texttt{Python rouge-score} package, which has inflated scores when compared to the \texttt{Perl} rouge script. Because we are looking at which hyper-parameter made the largest decrease, this is unimportant for the results of the test.} on this top model, to see which hyper-parameter causes the greatest increase in ROUGE score. We found that \textit{delta idf} for smoothing is the most important hyper-parameter for this system. 

Exp-ID J0 contains the best hyper-parameters, and has a ROUGE1 score of 0.33320. As we see in Exp-ID J5, we see that decreasing the \textit{delta idf} for smoothing to around 0, causes the greatest decrease in ROUGE1 score, (-0.10728). It also appears in Exp-ID J0, J2, and J3, that our choice of n-gram has an impact on the system with unigrams performing the best (-0.0), and trigrams performing the worst (-0.05638). Interestingly by Exp-ID J1, we found that discarding sentences under a certain sentence length has very little effect on the performance (-0.0812).

Likewise as we see in Exp-ID J4, whether choose a \textit{delta tf} that is close to 0 we get a decrease of (-0.0206) in ROUGE1 score, and thus conclude that \textit{delta tf} carries very little weight. In Exp-ID J6 through J7, whether we choose to eliminate all punctuation (-0.0), to log the \textit{tf-idf} values, or choose to lowercase all terms we see that these keep the ROUGE1 score the same (-0.0), and thus conclude that these hyper-parameters also had little to no effect on the performance of the system.

\subsubsection{Long Minimum Sentence Length}

In the exploration of hyper-parameters for our Integer Linear Programming content selection method, we found that the larger the minimum sentence length, the better the ROUGE1 and ROUGE2 score, finally settling on minimum sentence length equal to $25$. Interestingly, our ablation test found that sentence length matters little for the ROUGE score. This discrepancy between our initial exploration and our ablation test could rely on a couple of factors: the method with which we picked our best combination of hyper-parameters, as well as the fact that we are using \texttt{nltk.word\_tokenize} before counting the length of a sentence. 

Because we found the best combination of hyper-parameters using a manual coordinate descent, the order in which hyper-parameters are tested as well as which default hyper-parameters are chosen matters. We tested minimum sentence length first in our manual coordinate descent, and during these experiments found that increasing the minimum sentences length increased the ROUGE score. This correlation between increasing minimum sentence length and increasing ROUGE score makes sense, as longer sentences generally are more informative than shorter.

Also the minimum sentence length also is not what one would think of as an intuitive sentence length because \texttt{nltk.word\_tokenize} tends to inflate the length of a sentence. For example, "Columbine!" would have sentence length of $4$ because it would be tokenized into \texttt{[", Columbine, !, "]}. Thus this large minimum sentence length might be shorter than intuitive.

\subsubsection{$\delta_{idf}$ Creates ROUGE Score Jump}

For ILP, we observed that $\delta_{idf}$ creates the largest difference in ROUGE1 and ROUGE2 score. From this, we conclude that \textit{inverse document frequency (idf)} must have a greater impact on the importance of an n-gram than \textit{term frequency (tf)}. This result feels counter intuitive, namely that terms that appear more frequently must carry more weight. 

We do not know why $\delta_{idf}$ creates such a large jump in ROUGE score, but we suspect it might have to do with our choice not to remove stop words, and the way that $idf$ minimizes the importance of frequent words that appear in many documents. 

\subsection{NP-Replace}

The successes of the NP-replace algorithm are certainly diminished by the frequency of its failures. We suspect that because the unsupervised methods seem to favor longer, more descriptive sentences, the impact of replacing NPs is marginal to begin with. We also believe that many of the algorithm's failures can be attributed to its over-application--applying the algorithm to all NPs raises the likelihood of poor replacements. Perhaps the addition of an algorithm that can detect unseen references would improve its performance

\subsection{Ablation study}

The ablation study revealed that content realization algorithms have only marginal effects on the ROUGE scores (and quality upon visual examination) of the LexRank outputs. 

\subsection{Performance of Different Sentence Vectors}

For D4, we compared performance of LexRank using TF-IDF, DistilBERT, and Word2Vec vectors. Interestingly, the sentence vectors derived from neural network models performed worse than sparse vectors derived from TF-IDF values. This finding is surprising since neural networks have the potential to encode semantic information to some degree \cite{mikolov2013-word2vec}. 

We are unsure of why vectors derived from word or sentence embeddings perform worse. We speculate that, in the case of Word2Vec, calculating a centroid value from ``overwrites'' some of the semantic information in a sentence--especially if we weight each word equivalently. In the case of DistilBERT, we speculate that the dimensions of sentence embeddings can be arbitrary--they don't correspond to particular linguistic information--for out-of-the-box models. Perhaps the model would need to be finetuned using a contrastive loss function to better characterize sentence similarity. However, we are still largely unsure of why we observed these outcomes.

\subsection{Challenges}

We encountered a few hiccups while finalizing the system. For one, we experienced some difficulties combining the different information extraction, ordering, and realization methods results.

\subsection{Future Work}
For future work, we could continue to improve our content selection methods. For Integer Linear Programming, we could try a different concept weighting scheme. We can run more tests in search of better hyper-parameters, for example, we could run manual coordinate descent with a different default hyper-parameters in a different order. We could also try other ways to create concepts, such as skip-grams to try to capture similar concepts such as "the pandas" and "the giant pandas".

We could also improve our information ordering methods as well. For topic clustering, exploration is necessary on a different number of clusters other than 8. Too little clusters and one risks cramming the summary into the same space, too many clusters and one risks creating sparsity. Error analysis would then be necessary on the resulting summaries to find better topic clusters. 

Lastly, we could explore ways to overcome the input limits of the language model by using recursive summarization introduced by \citet{openaibook}, such as generate a summary for each of the document in a docset, and use those summaries to generate a final summary. 







%These are our future steps for the next step of the project:

%\begin{itemize}
%    \item Improve clustering by testing different K's for number of clusters.
%    \item Find a way to incorporate supervised learning techniques to leverage training data.
%    \item Possibly continue hyper-parameter tuning for ILP, LexRank, and large language model.
%    \item Investigate why the zero-shot information ordering produce undesired output 
%    \item Train the language model with other pre-trained models to compare the results.
%\end{itemize}

\section{Conclusion}

%The system created for D4 was a substantial improvement over D3. We successfully incorporated improvements to existing systems, and implemented new information ordering algorithms. Still, many of of our methods (such as ILP and LexRank) are unsupervised, meaning that we are leaving training data is going unused--a potential area for future exploration.


In conclusion, our paper explores multiple approaches to multi-document summarization, including both extractive and abstractive methods. We have built end-to-end systems using the selected methods and benchmarked them to evaluate their effectiveness. 
Our error analysis provides insights into the strengths and weaknesses of the selected methods, paving the way for future research on improving existing summarization methods and developing new summarization systems.

